\gdef\thesection{Appendix \Alph{section}}
\section{Mouvement des particules}
\label{sec:mouvement}
\begin{lstlisting}
void avancerParticules(unsigned int i)
{
	particules[i].tempsDeVieRestant -= deltaT;

	for (unsigned int j = 0 ; j < 3; j++)
	{
		// avancer la particule
		particules[i].position[j] += particules[i].vitesse[j] * deltaT;
	}
	// Mettre à jour la vitesse
	particules[i].vitesse[2] -= gravite * deltaT;
	
	for (unsigned int j = 0 ; j < 3; j++)
	{
		// vérifier les collisions
		if (particules[i].position[j] <= bMin[j] || particules[i].position[j] >= bMax[j])
		{
			particules[i].vitesse[j] = - particules[i].vitesse[j];
		}
	}
	
	if (particules[i].tempsDeVieRestant <= 0)
	{
		initPart(particules[i]);
	}
}
\end{lstlisting}

\gdef\thesection{Appendix \Alph{section}}
\section{Dépassement lors des collisions}
\label{sec:depassement}
\begin{figure}[h!tbp]
	\hspace*{-4.5em} \centering \includegraphics[width=0.90\paperwidth]{{depassement}.png}
	\centering \caption{\small Dépassement des particules lors des collisions}
	\label{fig:depassement}
\end{figure}

\gdef\thesection{Appendix \Alph{section}}
\section{Dépassement lors des collisions}
\label{sec:echappement}
\begin{figure}[h!tbp]
	\hspace*{-4.5em} \centering \includegraphics[width=0.90\paperwidth]{{echappement}.png}
	\centering \caption{\small Échappement de particules au niveau de la paroi supérieure}
	\label{fig:echappement}
\end{figure}