\documentclass[toc=left, paper=letterpaper, fontsize=12pt]{scrartcl}
\renewcommand{\baselinestretch}{1.5}
%\usepackage[T1]{fontenc}
\usepackage[utf8]{inputenc}
\usepackage[T1]{fontenc}
\usepackage{lmodern}
\usepackage{fourier}

\usepackage[english]{babel}                                                         % English language/hyphenation
\usepackage[protrusion=true,expansion=true]{microtype}  
\usepackage{amsmath,amsfonts,amsthm} % Math packages
\usepackage[pdftex]{graphicx}   
\usepackage{url}
\usepackage{hyperref}

\usepackage{listings}

\usepackage[nolist,nohyperlinks]{acronym}

\usepackage{xcolor}
\hypersetup{
	colorlinks,
	linkcolor={red!50!black},
	citecolor={blue!50!black},
	urlcolor={blue!80!black}
}

\usepackage{tcolorbox}

%%% Custom sectioning
\usepackage{sectsty}
\allsectionsfont{\centering \normalfont\scshape}


%%% Custom headers/footers (fancyhdr package)
\usepackage{fancyhdr}
\pagestyle{fancyplain}
\fancyhead{}                                            % No page header
\fancyfoot[L]{}                                         % Empty 
\fancyfoot[C]{}                                         % Empty
\fancyfoot[R]{\thepage}                                 % Pagenumbering
\renewcommand{\headrulewidth}{0pt}          % Remove header underlines
\renewcommand{\footrulewidth}{0pt}              % Remove footer underlines
\setlength{\headheight}{13.6pt}

%%% Equation and float numbering
\numberwithin{equation}{section}        % Equationnumbering: section.eq#
\numberwithin{figure}{section}          % Figurenumbering: section.fig#
\numberwithin{table}{section}               % Tablenumbering: section.tab#

%%% Code snippet
\lstset{
	language=C++, %% Troque para PHP, C, Java, etc... bash é o padrão
	basicstyle=\ttfamily\small,
	numberstyle=\footnotesize,
	numbers=left,
	backgroundcolor=\color{gray!10},
	frame=single,
	tabsize=2,
	rulecolor=\color{black!30},
	title=\lstname,
	escapeinside={\%*}{*)},
	breaklines=true,
	breakatwhitespace=true,
	framextopmargin=2pt,
	framexbottommargin=2pt,
	inputencoding=utf8,
	extendedchars=true,
	literate={á}{{\'a}}1 {ã}{{\~a}}1 {é}{{\'e}}1 {è}{{\`e}}1 {ç}{{\c{c}}}1 {à}{{\`a}}1,
}

%%% Maketitle metadata
\newcommand{\horrule}[1]{\rule{\linewidth}{#1}}     % Horizontal rule

%%% For todos
\usepackage[colorinlistoftodos]{todonotes}
\newcommand{\TODO}[2][]{\todo[inline,color=blue!10,size=\footnotesize,#1]{\textbf{TODO}: #2}}

%%% RESIF command
\usepackage{xspace}

\title{
        %\vspace{-1in}  
        \usefont{OT1}{bch}{b}{n}
        \normalfont \normalsize \textsc{École Polytechnique de Montréal} \\ [25pt]
        \includegraphics[scale=0.70]{poly} \\
        \vspace*{\fill}
        \horrule{0.5pt} \\[0.4cm]
        \huge INF2705 Infographie : Rapport technique \\
        \horrule{2pt} \\[0.5cm]
        \LARGE TP5 : Le système de particules
	    \vfill
}
\author{
        \normalfont                                 \normalsize
        \large  Alexandre Mao	\qquad	1813566\\[-3pt] \normalsize
        \large Maxime Schmitt	\qquad	1719088\\[-3pt]      \normalsize
        \today
}
\date{}


%%% Begin document
\begin{document}
\maketitle
%\clearpage

% ===================================
%\begin{abstract}
%	\input{_abstract}
%\end{abstract}
\clearpage

% ===================================
% acknowledgements ?

% Table of content
%\tableofcontents
%\clearpage

% ===================================
\section{Exposé du problème}
\label{sec:expose}
Le but du TP5, \textit{le système de particules}, est de modéliser l'évolution de particules soumises à un champ de gravité uniforme dans un univers fictif, délimité par des plans perpendiculaires, ainsi que permettre quelques interactions basiques avec celles-ci.
Ce TP se décompose en deux parties, la première ayant pour objectif l'implémentation des mouvements des particules dans l'espace considéré.
Pour déterminer les mouvements des particules au cours du temps, 2 aspects physiques sont à considérer : la gravité et les collisions, qui sont considérées comme des chocs rigides.
Initialement, les particules sont émises depuis un point fixe, que l'on nommera par la suite "puits", chacune étant créée avec une durée de vie limitée et déterminée aléatoirement dans un intervalle, un vecteur vitesse de direction et de norme aléatoire ainsi qu'une couleur.
On dispose également d'une image qui doit être utilisée comme un lutin pour l'affichage des particules.

Le second aspect de ce TP s'intéresse à la sélection graphique d'un objet pendant l'exécution, plus précisément à la sélection 3D par couleur.
Il doit être possible de sélectionner des particules en cliquant dessus dans la fenêtre.
Une fois sélectionnée, la particule concernée est affichée en noir pour qu'elle soit aisément reconnaissable à l'écran et puisse être réinitialisée, c'est-à-dire renvoyée au puits sous la forme d'une toute nouvelle particule.
Enfin, la sélection simultanée de multiples particules doit être possible, et dans ce cas la ré-initialisation doit s'appliquer à l'ensemble des particules sélectionnées.

% ===================================
\section{Ajouts ou modifications}
\label{sec:ajouts}
Les fichiers initiaux du TP nous fournissent une représentation de l'univers fictif avec l'affichage de 4 plans perpendiculaires pour modéliser les limites de celui-ci.
Dans la première partie du TP, nous avons modélisé des particules naissant à partir du puits se situant vers le centre de notre espace délimité par nos différents plans. 
Chaque particule se voit attribuer différentes caractéristiques initiales : une vitesse aléatoire en direction et en norme, une couleur fixée entre deux constantes \textit{COULMIN} et \textit{COULMAX}, une durée de vie déterminée aléatoirement entre 0 et \textit{tempsVieMax} secondes (initialement fixé à 5 secondes).
Nous utilisons 2 VAOs pour dessiner les parois de l'espace virtuel pour l'un et les particules pour l'autre.
Nous modélisons la gravité par l'ajout d'une force selon l'axe des z qui agit sur chaque particule.
La coordonnée selon l'axe z de chaque particule se voit donc, pour chaque unité de temps écoulée, soustraite l'effet de cette force par cette durée.
De cette façon chaque particule décrit un mouvement correspondant à une chute libre.
Elles sont de plus soumises à des collisions avec les parois de l'espace fictif, qui délimite le volume dans lequel elles peuvent évoluer, sous la forme de chocs rigides.
Pour implémenter cette fonctionnalité, lorsqu'on détecte une collision sous la forme de la présence d'une particule en dehors du volume autorisé, on effectue la réflexion du vecteur vitesse de celle-ci par rapport au plan traversé:
De cette façon, la particule effectue son retour dans l'espace délimité lors de la prochaine itération et ce sans perte d'énergie.
Pour afficher les particules à l'aide de lutins, nous créons dans le nuanceur de géométrie un carré autour du point représentant chacune d'elles afin d'y appliquer une texture dans le nuanceur de fragment et n'afficher à l'écran que la zone colorée.

Dans la seconde partie, nous implémentons la sélection 3D  par couleur.
Pour cela, nous fournissons à chaque particules un identifiant unique sous la forme d'une couleur que l'on n'affichera pas mais qui permettra leur distinction lorsque l'on effectuera un clic de souris pour en sélectionner une.
Cet identifiant est donc codé sur trois octets (les trois octets du code de couleur RGB), ce qui permet un maximum de \begin{math} 2^{24} \end{math} particules uniques.
Chaque particule se voit attribué cet identifiant unique lors de sa création initiale : la \textit{ième} particule créée se voit attribuer l'identifiant \textit{(i mod 255, (i/255) mod 255, (i/\begin{math} 255^{2} \end{math}) mod 255)}.
En particulier, lorsqu'une particule est réinitialisée, son identifiant n'est pas modifié.
La ré-initialisation, qu'elle soit manuelle ou due à la fin de vie de la particule, consiste donc en la génération de nouvelles valeurs pour toutes ses caractéristiques hormis celle liée à la sélection.

%====================================
\section{Discussion}
\label{sec:discussion}
Lors de ce TP, nous avons pu mettre en pratique certains éléments vus lors des cours, dont le premier est celui de l'usage des VAOs et VBOs.
Nous avons utilisé deux VAOs pour dessiner les plans limitant l'espace accessible aux particules pour l'un et les particules elles-mêmes pour l'autre, optimisant de ce fait l'affichage en tirant profit des optimisations internes d'OpenGL lors du chargement de l'un et l'autre comparé à des chargements de VBOs dans un unique VBO.
Dans ce second VAO, nous avons fait le choix d'utiliser l'option \textit{GL\_STREAM\_DRAW} dans la fonction \textit{glBufferData} pour le VBO contenant les informations sur les particules.
En effet, les données dans ce VBO ne sont utilisées qu'une seule fois avant d'être écrasées, rendant ce choix d'option particulièrement adapté à la situation.

De son côté, la fonction d'initialisation des particules \textit{initPart} génère des valeurs initiales pour la direction et la valeur de la vitesse initiale, la couleur d'affichage et la durée de vie d'une particule tout en la plaçant au puits.
Nous l'utilisons donc en particulier pour l'initialisation et la ré-initialisation des particules pour mettre en place le comportement demandé par les exigences.

Concernant la représentation de la gravité, la modélisation que nous utilisons nous permet d'avoir une représentation très basique du phénomène de la collision par choc rigide et l'ajout de la force gravitationnelle selon la composante des z donne bien une trajectoire parabolique aux particules.
Mais dans l'implémentation initial (voir~\ref{sec:mouvement}), la collision est détectée lorsqu'une particule est déjà sortie du volume dans lequel elle est censée être contrainte.
Une fois cette situation détectée on inverse simplement sa vitesse après l'avoir actualisée ce qui permet d'assurer le retour de la particule dans le volume lors de la prochaine itération.
Cependant, cela n'est pas toujours le cas lorsque l'on s'intéresse aux collisions avec la paroi supérieure de l'espace considéré.
En effet, comme on actualise la vitesse avant de l'inverser, il existe un cas où la nouvelle vitesse de la particule ainsi obtenue n'est pas suffisante pour retraverser la paroi à la prochaine itération.
La particule est alors bloquée à l'extérieur du volume, s'en éloignant progressivement, jusqu'à sa fin de vie, sa vitesse s'inversant de sens à chaque itération mais en ayant toujours une norme plus grande lors de l'éloignement de la surface que lors de son rapprochement.
Notre représentation des collisions est donc imparfaite tant parce qu'elle laisse les particules traverser les surfaces avant de les "renvoyer dans le volume" (voir figure~\ref{fig:depassement}) que parce qu'elle présente un effet de bord laissant certaines particules "s'échapper" du volume (voir figure~\ref{fig:echappement}).
C'est pourquoi l'une des améliorations qui pourrait être faite à ce niveau est de mieux prendre en compte le choc rigide, en faisant en sorte que la particule ne dépasse pas les limites de l'univers, nous pourrions par exemple les faire s'arrêter au niveau du plan, cela entraînerait une sorte de compression des vitesses à proximité des parois, ou calculer la position suivant la collision mais dans ce cas aussi, nous aurions une sorte de trajectoire étrange pour la particule qui semblerait rebondir avant le bord de l'univers, même si cette sensation pourrait en fait être masquée par les dimensions des particules s'étendant au-delà du point considéré pour représenter leur position.
De plus, cela se ferait au prix de davantage d'opérations à effectuer pour traiter le cas des collisions : ce que l'on gagnerait en précision et justesse serait alors perdu en performance et donc en capacité maximal de particules que pourrait supporter le programme.

Lors de la représentation des particules sous formes de lutin, on utilise dans un premier temps le nuanceur de géométrie pour dessiner un carré autour du point de coordonnée de chaque particule dont la face avant se retrouve toujours en face de l'observateur.
C'est sur ce carré que, après avoir chargé et transmis au nuanceur de fragment la texture souhaitée, nous interpolons la couleur en fonction de la texture appliquée et de la couleur de la particule puis supprimons les parties dont la somme des couleurs est inférieure à un seuil de 1.
Cela correspond aux zones les plus sombres, et donne donc aux particules la forme voulue, le noir dans la texture correspondant à "l'extérieur de la forme".
Cette approche permet de donner une impression de 3D aux particules de façon peu coûteuse, mais donne une illusion relativement faible que l'on met facilement en évidence en observant plus attentivement le rendu. Une amélioration du rendu, au prix d'un coût en performance, serait d'utiliser la tesselation comme vue en cours afin de dessiner un volume autour des particules au lieu d'un plan et d'y appliquer ensuite une texture plus complexe sur toute la surface ainsi créée.
Le résultat serait alors bien plus convaincant mais le nombre de particules maximal devrait certainement être revu à la baisse puisque le nombre de points à manipuler pour chacune d'elles deviendrait bien plus important que les 4 coins du carré actuel.

Comme décrit dans la partie précédente, chaque particule se voit attribuer une couleur unique, qui n'est pas affichée, afin de permettre la sélection. Le nombre maximal de \begin{math}2^24\end{math} particules distinctes possibles avec notre méthode est largement suffisant puisque bien supérieur aux nombre maximum de 100 000 particules prévu dans le TP et même bien supérieur à la capacité d'affichage de nos machines, le programme n'étant déjà plus fluide avec 1 000 000 de particules.
Lors d'un clic à l'écran sur une particule, nous la retrouvons parmi toutes les autres particules créées grâce à cet identifiant unique, la marquons comme sélectionnée (à l'aide de sa variable d'état \textit{estSelectionne}) et l'affichons alors en noir.
Toutes les particules sélectionnées peuvent alors être réinitialisées, en utilisant la touche \textit{p}, grâce à la fonction \textit{ramenerSelectionnes} qui se charge de parcourir l'ensemble des particules, et de réinitialiser chaque particule dont l'état \textit{estSelectionnee} est vrai.
Cette solution répond aux exigences du TP et est suffisamment efficace dans son cadre, mais il est possible que la tenue d'une table des particules sélectionnées puisse permettre une optimisation du programme, notamment lorsqu'on considère le cas d'un grand nombre de particules, puisque cela permettrait d'éviter certains des parcours de l'ensemble des particules existantes, en particulier lors de leur ré-initialisation.

%====================================
\section{Difficultés rencontrées}
\label{sec:difficultes}
Lors du développement nous avons rencontrées divers difficultés concernant la représentation de certaines points spécifiques du TP.

La première d'entre elles est liée à la gestion des collisions avec les parois.
En effet, en fonction du choix d'actualiser la vitesse avant ou après son inversion suite à la collision, on a un effet indésirable de particules bloquées en dessous du volume si on le fait avant et au dessus si on le fait après, comme expliqué dans la partie précédente.
Comme ce second cas est plus rare et nécessite des conditions particulières pour être observé (gravité forte et durée de vie longue pour les particules), nous avons fait le choix d'utiliser cette alternative.
Cependant, une véritable solution à ce problème serait de changer le traitement des collisions comme proposé dans la partie précédente.

Dans la seconde partie du TP, nous avions utilisé l'idée simple lors de la sélection d'une particule, de l'identifier grâce à sa couleur d'affichage, attribuée aléatoirement lors de son initialisation.
Cependant nous avons remarqué que non seulement cette couleur n'était pas unique mais également qu'une particule affichée à l'écran n'était pas constituée d'une seule et unique couleur à cause de l'interpolation de la couleur dans le nuanceur de fragment au moment d'appliquer la texture.
Pour résoudre ce problème, nous avons dû attribuer une couleur unique non affichée à chaque particule, comme expliqué dans la partie précédente, puis nous traitions la sélection en considérant le tampon arrière non affiché dans lequel on traçait les particules avec leur couleur unique.
Cette implémentation de la sélection nous permettait alors bien de discerner toutes les particules et ce pour un nombre de particules allant jusqu'à plus de 16 millions, nombre bien supérieur à ce que l'on considérait en pratique.




% ===================================
\section{Conclusion}
\label{sec:conclusion}
D'une façon générale, le programme développé au cours de ce TP permet la représentation de l'évolution d'un grand nombre de particules soumises à un champ gravitationnel dans un volume fermé grâce à un rendu simpliste tant visuellement que du point de vue des lois de la physique.
On pourrait l'améliorer pour rendre le rendu visuel crédible et les interactions physiques plus réaliste, en prenant par exemple en compte des forces de frottement ou les chocs entre particules, mais cela se ferait logiquement au détriment du nombre maximal de particules affichables.

\clearpage
% \singlespacing

\appendix
\gdef\thesection{Appendix \Alph{section}}
\section{Mouvement des particules}
\label{sec:mouvement}
\begin{lstlisting}
void avancerParticules(unsigned int i)
{
	particules[i].tempsDeVieRestant -= deltaT;

	for (unsigned int j = 0 ; j < 3; j++)
	{
		// avancer la particule
		particules[i].position[j] += particules[i].vitesse[j] * deltaT;
	}
	// Mettre à jour la vitesse
	particules[i].vitesse[2] -= gravite * deltaT;
	
	for (unsigned int j = 0 ; j < 3; j++)
	{
		// vérifier les collisions
		if (particules[i].position[j] <= bMin[j] || particules[i].position[j] >= bMax[j])
		{
			particules[i].vitesse[j] = - particules[i].vitesse[j];
		}
	}
	
	if (particules[i].tempsDeVieRestant <= 0)
	{
		initPart(particules[i]);
	}
}
\end{lstlisting}

\gdef\thesection{Appendix \Alph{section}}
\section{Dépassement lors des collisions}
\label{sec:depassement}
\begin{figure}[h!tbp]
	\hspace*{-4.5em} \centering \includegraphics[width=0.90\paperwidth]{{depassement}.png}
	\centering \caption{\small Dépassement des particules lors des collisions}
	\label{fig:depassement}
\end{figure}

\gdef\thesection{Appendix \Alph{section}}
\section{Dépassement lors des collisions}
\label{sec:echappement}
\begin{figure}[h!tbp]
	\hspace*{-4.5em} \centering \includegraphics[width=0.90\paperwidth]{{echappement}.png}
	\centering \caption{\small Échappement de particules au niveau de la paroi supérieure}
	\label{fig:echappement}
\end{figure}

\clearpage

\bibliographystyle{IEEEtran}
\bibliography{biblio}

% Available acronyms
\input{_acronyms}


\end{document}
