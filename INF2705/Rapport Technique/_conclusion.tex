D'une façon générale, le programme développé au cours de ce TP permet la représentation de l'évolution d'un grand nombre de particules soumises à un champ gravitationnel dans un volume fermé grâce à un rendu simpliste tant visuellement que du point de vue des lois de la physique.
On pourrait l'améliorer pour rendre le rendu visuel crédible et les interactions physiques plus réaliste, en prenant par exemple en compte des forces de frottement ou les chocs entre particules, mais cela se ferait logiquement au détriment du nombre maximal de particules affichables.