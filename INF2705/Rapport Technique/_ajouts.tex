Les fichiers initiaux du TP nous fournissent une représentation de l'univers fictif avec l'affichage de 4 plans perpendiculaires pour modéliser les limites de celui-ci.
Dans la première partie du TP, nous avons modélisé des particules naissant à partir du puits se situant vers le centre de notre espace délimité par nos différents plans. 
Chaque particule se voit attribuer différentes caractéristiques initiales : une vitesse aléatoire en direction et en norme, une couleur fixée entre deux constantes \textit{COULMIN} et \textit{COULMAX}, une durée de vie déterminée aléatoirement entre 0 et \textit{tempsVieMax} secondes (initialement fixé à 5 secondes).
Nous utilisons 2 VAOs pour dessiner les parois de l'espace virtuel pour l'un et les particules pour l'autre.
Nous modélisons la gravité par l'ajout d'une force selon l'axe des z qui agit sur chaque particule.
La coordonnée selon l'axe z de chaque particule se voit donc, pour chaque unité de temps écoulée, soustraite l'effet de cette force par cette durée.
De cette façon chaque particule décrit un mouvement correspondant à une chute libre.
Elles sont de plus soumises à des collisions avec les parois de l'espace fictif, qui délimite le volume dans lequel elles peuvent évoluer, sous la forme de chocs rigides.
Pour implémenter cette fonctionnalité, lorsqu'on détecte une collision sous la forme de la présence d'une particule en dehors du volume autorisé, on effectue la réflexion du vecteur vitesse de celle-ci par rapport au plan traversé:
De cette façon, la particule effectue son retour dans l'espace délimité lors de la prochaine itération et ce sans perte d'énergie.
Pour afficher les particules à l'aide de lutins, nous créons dans le nuanceur de géométrie un carré autour du point représentant chacune d'elles afin d'y appliquer une texture dans le nuanceur de fragment et n'afficher à l'écran que la zone colorée.

Dans la seconde partie, nous implémentons la sélection 3D  par couleur.
Pour cela, nous fournissons à chaque particules un identifiant unique sous la forme d'une couleur que l'on n'affichera pas mais qui permettra leur distinction lorsque l'on effectuera un clic de souris pour en sélectionner une.
Cet identifiant est donc codé sur trois octets (les trois octets du code de couleur RGB), ce qui permet un maximum de \begin{math} 2^{24} \end{math} particules uniques.
Chaque particule se voit attribué cet identifiant unique lors de sa création initiale : la \textit{ième} particule créée se voit attribuer l'identifiant \textit{(i mod 255, (i/255) mod 255, (i/\begin{math} 255^{2} \end{math}) mod 255)}.
En particulier, lorsqu'une particule est réinitialisée, son identifiant n'est pas modifié.
La ré-initialisation, qu'elle soit manuelle ou due à la fin de vie de la particule, consiste donc en la génération de nouvelles valeurs pour toutes ses caractéristiques hormis celle liée à la sélection.